\documentclass[8pt]{extarticle}
\usepackage[english]{babel}
\usepackage[utf8]{inputenc}
\usepackage{xcolor}
\usepackage[a4paper,left=2.3cm,right=1.2cm,top=2cm,bottom=2cm]{geometry} 
\usepackage{blindtext}
\usepackage{setspace}
\usepackage{float}
\usepackage{titletoc}
\usepackage{titlesec}
\usepackage{wrapfig}
\usepackage{tikz}
\usepackage{amsmath} 
\usepackage{multicol}
\usepackage{amsfonts} 
\usepackage{comment}
\usepackage{booktabs}
\usepackage{bbm}
\usepackage{wrapfig}
\usepackage{verbatimbox}
\usepackage{enumitem}
\usepackage[framemethod=TikZ]{mdframed}
\usepackage{bigints}
\onehalfspacing
\usepackage[hidelinks]{hyperref}
\usepackage[all]{nowidow} %funktioniert nicht....
\clubpenalty=9996
\widowpenalty=9999
\brokenpenalty=4991
\predisplaypenalty=10000
\postdisplaypenalty=1549
\displaywidowpenalty=1602
\usepackage[round]{natbib} 
\usepackage{enumitem}

\usepackage{titlesec}
\newcommand{\sectionbreak}{\clearpage}

\allowdisplaybreaks

\setlength\parindent{0pt}

\newcommand{\zerodisplayskips}{%
  \setlength{\abovedisplayskip}{2pt}%
  \setlength{\belowdisplayskip}{2pt}%
  \setlength{\abovedisplayshortskip}{2pt}%
  \setlength{\belowdisplayshortskip}{2pt}}
\appto{\normalsize}{\zerodisplayskips}
\appto{\small}{\zerodisplayskips}
\appto{\footnotesize}{\zerodisplayskips}

\newcommand\independent{\protect\mathpalette{\protect\independenT}{\perp}}
\def\independenT#1#2{\mathrel{\rlap{$#1#2$}\mkern2mu{#1#2}}}



%Hier sind die unterschiedlichen Ausführlichkeitsgrade definiert
\includecomment{Extensiv} 
\includecomment{Proof} 
\includecomment{Annahmen}
\includecomment{Mathspez}
\includecomment{Mathfolg}
\includecomment{Rechreg}
\mdfdefinestyle{MyFrame}{%
    linecolor=black!20!,
    outerlinewidth=0.2pt,
    roundcorner=5pt,
    innertopmargin=0.5\baselineskip,
    innerbottommargin=0.5\baselineskip,
    innerrightmargin=10pt,
    innerleftmargin=10pt,
    backgroundcolor=white}
\specialcomment{Proof}{\begin{mdframed}[style=MyFrame,nobreak=true] Proof: \ \\}{\end{mdframed}}
\specialcomment{Rechreg}{\noindent \textit{Calculation Rules:} \begin{itemize}[nosep,label=$\star$] }{\end{itemize}}
\renewcommand\ThisComment[1]{% Fix for Umlauts in comments
  \immediate\write\CommentStream{\unexpanded{#1}}%
}

% Hier die Ausführlichkeit bestimmen:
%\excludecomment{Extensiv} 
%\excludecomment{Proof} 
%\excludecomment{Annahmen}
%\excludecomment{Mathspez}
%\excludecomment{Mathfolg}

% Inhaltsverzeichnis mit zwei Spalten
\usepackage[toc]{multitoc}
\renewcommand*{\multicolumntoc}{2}




%Überschriftengrößen anpassen, so dass Paragraph kleiner ist als Subsubsection
\titleformat{\section}
  {\normalfont\fontsize{16}{15}\bfseries}{\thesection}{1em}{}
\titleformat{\subsection}
  {\normalfont\fontsize{14}{15}\bfseries}{\thesubsection}{1em}{}
\titleformat{\subsubsection}
  {\normalfont\fontsize{12}{15}\bfseries}{\thesubsubsection}{1em}{}


\begin{document}

\topskip0pt
\vspace*{18em}

\hrule
\begin{center}
{\fontsize{30}{60}\selectfont \textbf{Causal Inference}} \\ \

{\fontsize{20}{60}\selectfont a summary}
\end{center}
\hrule

\tableofcontents




% weitere Anpassungen im Hauptteil des Dokuments
\raggedright %linksbündig
\setlength{\parindent}{15pt} %Einzuglänge festsetzen
\setlength{\columnseprule}{0.3pt} %Liniendicke zwischen zwei Multicols






%-------------------------------------------------------------------------------

% SECTION: GENERAL

%-------------------------------------------------------------------------------

\section{General}

\begin{multicols}{2}

\paragraph{\large Causal Roadmap} \citep{petersen2014causal} 
systematic approach linking causality to statistical procedures

\noindent \textbf{1. Specifying Knowledge.} structural causal model (unifying counterfactual language, structural equations, \& causal graphs): a set of possible data-generating processes, expresses background knowledge and its limits

\noindent \textbf{2. Linking Data.} specifying measured variables and sampling specifics (latter can be incorporated into the model)

\noindent \textbf{3. Specifying Target.} define hypothetical experiment: decide
\noindent\begin{enumerate}[itemsep=0em, topsep=0pt, partopsep=0pt,parsep=0pt]
\setlength{\itemsep}{0pt}%
\setlength{\parskip}{0pt}
\item variables to intervene on: one (point treatment), multiple (longitudinal, censoring/missing, (in)direct effects)
\item intervention scheme: static, dynamic, stochastic
\item counterfactual summary of interest: absolute or relative, marginal structural models, interaction, effect modification
\item population of interest: whole, subset, different population
\end{enumerate}

\noindent \textbf{4. Assessing Identifiability.} are knowledge and data sufficient to derive estimand and if not, what else is needed?

\noindent \textbf{5. Select Estimand.} current best answer: knowledge-based assumptions $+$ which minimal convenience-based asspumptions (transparency) gets as close as possible

\noindent \textbf{6. Estimate.} choose estimator by statistical properties, nothing causal here

\noindent \textbf{7. Interpret.} hierarchy: statistical, counterfactual, feasible intervention, randomized trial



\paragraph{Notation} chapter 1.1

\paragraph{average causal effect} chapter 1.2 and 1.3 and 1.4 and 1.5

\paragraph{randomized experiments (target trial)} 2.1 and 2.2; 3.6

\paragraph{standardization} 2.3

\paragraph{\large IP Weighting} 2.4 (adjust for surrogate confounders)





\paragraph{identifiability conditions} most of 3

\paragraph{effect modification} chapter 4

\paragraph{interaction} chapter 5

\paragraph{causal diagrams} chapter 6, include swigs from 7.5 and that one technical point

\paragraph{confounding} chapter 7

\paragraph{selection bias} chapter 8

\paragraph{measurement bias} chapter 9

\paragraph{random variabilty} chapter 10


\end{multicols}



%-------------------------------------------------------------------------------

% SECTION: MODELS

%-------------------------------------------------------------------------------

\section{Models}
\begin{multicols}{2}

\paragraph{\large Modeling} data are a sample from the target population \vspace{0.4em}

\noindent \hspace{0.9em}\begin{tabular}{l l l}
\textbf{\it estimand:} & quantity  of interest, & e.\,g. $\mathrm{E}\left[Y|A=a\right]$ \\
\textbf{\it estimator:} & function to use, & e.\,g. $\widehat{\mathrm{E}}\left[Y|A=a\right]$ \\
\textbf{\it estimate:} & apply function to data, & e.\,g. $4.1$ 
\end{tabular} \vspace{0.5em}

\noindent \textbf{model}: a priori restriction of joint distribution/dose-response curve;  \textit{assumption:} no model misspecification (usually wrong)

\noindent \textbf{non-parametric estimator:} no restriction (saturated model) $=$ \textit{Fisher consistent estimator} (entire population data $\rightarrow$ true value)

\noindent \textbf{parsimonious model:} few parameters estimate many quantities

\noindent \textbf{bias-variance trade-off:} \newline wiggliness $\uparrow$ $\rightarrow$ misspecification bias $\downarrow$, CI width $\uparrow$


\paragraph{\large IP Weighting}  


/ marginal structural models

(comparison at 13.4) (maybe censoring as own paragraph)



\paragraph{standardization / parametric g-formula} chapter 13

\paragraph{g-estimation} chapter 14

\paragraph{Outcome regression} chapter 15

\paragraph{instrumental variable estimation} chapter 16

\paragraph{causal survival analysis} chapter 17

\paragraph{variable selection} beginning of chapter 18

\paragraph{machine learning in CI} end of chapter 18




\end{multicols}

%-------------------------------------------------------------------------------

% SECTION: LONGITUDINAL

%-------------------------------------------------------------------------------

\section{Longitudinal Data}
\begin{multicols}{2}

\paragraph{time-varying treatments} beginning chapter 19

\paragraph{identifiability} middle chapter 19

\paragraph{treatment-confounder feedback} end chapter 19 and chapter 20

\paragraph{g-formula} chapter 21.1

\paragraph{IP weighting} chapter 21.2

\paragraph{doubly robust estimators} chapter 21.3

\paragraph{g-estimation} chapter 21.4

\paragraph{censoring} chapter 21.5

\paragraph{target trial} chapter 22 (does that even really fit in here, maybe push to 3rd paragraph in without models)



\end{multicols}


\def\bibpreamble{\textit{If no citation is given, the source is \citep{hernan2023causal}} \vspace{1.5em}}

\bibliographystyle{apalike} % We choose the "plain" reference style
\bibliography{cite} % Entries are in the refs.bib file



\end{document}